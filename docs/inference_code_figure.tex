\begin{figure}[t]
\begin{minipage}[t]{0.55\textwidth}
\begin{lstlisting}
function importance_sample(input_image, num_samples)
  observations = Trace()
  observations[<@\addr{"image"}@>] = input_image

  # obtain importance samples and their weights
  traces = Vector{Trace}(num_samples)
  log_weights = Vector{Float64}(num_samples)
  for i=1:num_samples
    (traces[i], _, log_weights[i], _) = <@\infr{importance2}@>(
      model, (), proposal, (), observations)
  end

  # return trace with probability proportional to its weight
  probs = exp.(log_weights - logsumexp(log_weights))
  idx = rand(categorical, (probs,))
  return traces[idx]
end
\end{lstlisting}
\end{minipage}
\caption{
Performing inference in the generative model of Figure~\ref{fig:model-code-figure} using a combination of deep learning and model-based Monte Carlo.
Having trained the proposal distribution (\texttt{proposal}) we use it as an importance distribution in a sampling-importance-resampling algorithm.
The GenLite API function \infr{\texttt{importance2}} samples from an importance distribution by simulating from a generative function, and then treats the resulting trace as a proposed trace for the model.
}
\label{fig:inference-code-figure}
\end{figure}
