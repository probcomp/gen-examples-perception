\section{Scalable Deep Learning with TensorFlow Integration}

% language (functional subset of TensorFlow with @param)

% implementation (hybrid AD)

Although it is possible to implement deep neural networks in Julia and invoke these from generative functions using static parameters to store network weights, support for GPU and deep learning in Julia is still new.
Therefore, GenLite includes the option of using TensorFlow for deep learning.
We extend GenLite with a \texttt{tf\_function} and \texttt{tf\_call} language constructs.
The \texttt{tf\_function} keyword is used to define \emph{GenLite TensorFlow functions}, which are functional TensorFlow computations with declared inputs, trainable parameters, and an output, that can be invoked by generative functions (see Figure~\ref{fig:proposal-code-figure} for an example).
Within \texttt{tf\_function} blocks, we construct TensorFlow computations using the TensorFlow.jl \cite{?} wrapper around the TensorFlow C API, as well as three new keywords that facilitate integration with GenLite:
\begin{enumerate}
\item \texttt{@input <name> <dtype> <shape>}:
Assigns a TF placeholder with given data type and shape to variable \texttt{name}, and registers this with GenLite as an input of the TF function.
\item \texttt{@param <name> <initial-julia-value>}:
Declare a static parameter for the TensorFlow function.
Assigns a TF variable with given initial value variable \texttt{name}, and registers this with GenLite as an input of the TF function.
GenLite constructs an additional variable to store the accumulated gradient of output with respect to the variable \texttt{<name>}.
\item \texttt{@output <dtype> <tensor-value>}:
Registers the given tensor value as the output of the TF function and statically declares its data type.
\end{enumerate}
Note that static parameters of TensorFlow functions have similar semantics to static parameters of generative functions, but are implemented differently (e.g. the values are owned by the TensorFlow runtime instead of Julia) and are accessed and modified using a different interface (a TensorFlow interface as opposed to a Julia interface).

TensorFlow functions are invoked by generative functions using:
\begin{center}
    \texttt{@tf\_call(<tf-function>([input1, [input2, ..]]))}
\end{center}
where each input is a Julia value corresponding to an \texttt{@input} declaration (in the order of the declarations in the body of the function).
Evaluating a \texttt{@tf\_call} expression invokes the TF computation and returns the tensor registered as \texttt{@output} as a Julia \texttt{Array} value.
When the expression is evaluated in GenLite's \texttt{backprop} API method, evaluating the \texttt{@tf\_call} expression also causes a cell for the TF function to be placed on the GenLite's AD tape (see Figure~\ref{fig:tf-integration-schematic}).
During the backward pass of reverse-mode AD, the gradient with respect to \texttt{@output} is used to increment the gradient with respect to the inputs, by invoking the appropriate TensorFlow gradient computation.
The gradient with respect to the registered parameters are also computed in TensorFlow, but instead of being passed back into Julia, the parameter gradient values are used to increment TensorFlow gradient accumulator variables.

The user writes TensorFlow code to update the parameters of a TensorFlow function based on the accumulated gradients.
The update code is defined separately to the \texttt{@tf\_function} definition, to preserve the functional semantics of the TensorFlow function.
GenLite provides methods that give access the TF variables for the parameters and their gradients for a given TensorFlow function (e.g. \texttt{get\_param\_var} and \texttt{get\_param\_grad}) see Figure~\ref{fig:training-code-figure}(a)).
The update code is responsible for updating the variables and resetting the gradients to zero (a helper function \texttt{get\_zero\_grad\_op} is provided for this).
Note that Tensorflow parameters and gradients are not copied between the TensorFlow and Julia runtimes during either backpropagation or update.

Finally, note that because gradients may be accumulated over multiple backpropagation passes, users have the option of performing batch parameter optimization without writing vectorized (i.e. batched) code.
Of course, users may also write batched TensorFlow code, in which case executions of the parameter update are interleaved with executions of \texttt{backprop}.

\begin{figure}[h]
\centering
    \includegraphics[width=0.7\textwidth]{images/tf-integration-schematic.pdf}
    \caption{
Reverse-mode AD in GenLite interoperates with `GenLite TensorFlow functions', which are blocks of functional TensorFlow (TF) code with inputs (corresponding to TF placeholders), trainable parameters (corresponding to TF variables), and an output, that are invoked by GenLite generative functions.
Each invocation of a TF function produces a single element on GenLite's reverse-mode AD tape.
During the backward pass (solid lines), we receive the gradient with respect to the output (\texttt{@output}) of the TF function; TF is used to compute the gradients with respect to inputs (\texttt{@input}) and parameters (\texttt{@params}).
The gradients with respect to the parameters are accumulated across multiple backward passes, until an parameter update is performed.
A parameter update (dashed line) changes the parameter values using the accumulated gradient (in addition to state of the update operation itself), and resets the gradient accumulators to zero.
Parameter updates are TF computations that are defined by the user separately from the TF function itself.
}
    \label{fig:tf-integration-schematic}
\end{figure}

\begin{figure}[t]
\begin{minipage}[t]{0.5\textwidth}
\begin{lstlisting}
using Cairo, ImageMagick, ImageFiltering
..

function render(glyph::Glyph)
  canvas = CairoRGBSurface(width, height)
  cc = CairoContext(canvas)
  Cairo.save(cc)

  # set background color to white
  Cairo.set_source_rgb(cc, 1.0, 1.0, 1.0)
  Cairo.rectangle(cc, 0.0, 0.0, width, height)
  Cairo.fill(cc)
  Cairo.restore(cc)
  Cairo.save(cc)

  # write the letter
  fontface = "Sans $(glyph.fontsize)"
  Cairo.set_font_face(cc, fontface)
  Cairo.text(cc, glyph.x, glyph.y, glyph.letter,
             angle=glyph.angle)

  return convert_to_png_blob(canvas)
end
\end{lstlisting}
\end{minipage}%
\begin{minipage}[t]{0.5\textwidth}
\begin{lstlisting}
model = @gen function()
  # prior
  x = width * @rand(uniform_cont(0, 1), <@\addr{"x"}@>)
  y = height * @rand(uniform_cont(0, 1), <@\addr{"y"}@>)
  size = @rand(uniform_cont(0, 1), <@\addr{"size"}@>)
  letter_id = @rand(uniform_disc(1, 3), <@\addr{"letter"}@>)
  letter = ["A", "B", "C"][letter_id]
  angle = 45 * @rand(uniform_cont(-1, 1), <@\addr{"angle"}@>)
  fontsize = scale_size(min_size, max_size, size)
  glyph = Glyph(x, y, angle, fontsize, letter)

  # render to png bytes
  image_png = render(glyph)

  # add Gaussian blur
  blur_width = 3
  blurred_png = imfilter(image_png,
                  Kernel.gaussian(blur_width))

  # add noise
  matrix = convert(Matrix{Float64}, blurred_png)
  @rand(speckle_noise(matrix, 0.1), <@\addr{"image"}@>)
end
\end{lstlisting}
\end{minipage}
\caption{Generative function for a generative model of blurry images that contain a single letter at a random location, rotation, and size. Addresses of random choices are shown in green.}
\label{fig:model-code-figure}
\end{figure}

\begin{figure}[t]
\begin{minipage}[t]{0.6\textwidth}
\begin{lstlisting}
using GenLiteTF
using TensorFlow
tf = TensorFlow

num_input = width * height
num_output = 11

function conv2d(x, W)
  tf.nn.conv2d(x, W, [1, 1, 1, 1], "SAME")
end

function max_pool_2x2(x)
  tf.nn.max_pool(x, [1, 2, 2, 1], [1, 2, 2, 1], "SAME")
end

function initial_weight(shape)
  randn(Float32, shape...) * 0.001f0
end

function initial_bias(shape)
  fill(0.1f0, shape...)
end

network = @tf_function begin

  # input image (N, 56 * 56)
  @input image_flat Float32 [-1, num_input]
  image = tf.reshape(image_flat, [-1, width, height, 1])

  # convolution + max-pooling (N, 28, 28, 32)
  @param W_conv1 initial_weight([5, 5, 1, 32])
  @param b_conv1 initial_bias([32])
  h_conv1 = tf.nn.relu(conv2d(image, W_conv1) + b_conv1)
  h_pool1 = max_pool_2x2(h_conv1)

  # convolution + max-pooling (N, 14, 14, 32)
  @param W_conv2 initial_weight([5, 5, 32, 32])
  @param b_conv2 initial_bias([32])
  h_conv2 = tf.nn.relu(conv2d(h_pool1, W_conv2) + b_conv2)
  h_pool2 = max_pool_2x2(h_conv2)
  h_pool2_flat = tf.reshape(h_pool2, [-1, 14 * 14 * 32])

  # convolution + max-pooling (N, 7, 7, 64)
  @param W_conv3 initial_weight([5, 5, 32, 64])
  @param b_conv3 initial_bias([64])
  h_conv3 = tf.nn.relu(conv2d(h_pool2, W_conv3) + b_conv3)
  h_pool3 = max_pool_2x2(h_conv3)
  h_pool3_flat = tf.reshape(h_pool3, [-1, 7 * 7 * 64])

  # fully connected layer (N, 1024)
  @param W_fc1 initial_weight([7 * 7 * 64, 1024])
  @param b_fc1 initial_bias([1024])
  h_fc1 = tf.nn.relu(h_pool3_flat * W_fc1 + b_fc1)

  # output layer (N, 11)
  @param W_fc2 initial_weight([1024, num_output])
  @param b_fc2 initial_bias([num_output])
  @output Float32 (tf.matmul(h_fc1, W_fc2) + b_fc2)
end

\end{lstlisting}
\end{minipage}%
\begin{minipage}[t]{0.4\textwidth}
\begin{lstlisting}
predict = @gen function (outputs)

  # predict the x-coordinate
  x_mu = outputs[1]
  x_std = exp(outputs[2])
  @rand(normal(x_mu, x_std), <@\addr{"x"}@>)

  # predict the y-coordinate
  y_mu = outputs[3]
  y_std = exp(outputs[4])
  @rand(normal(y_mu, y_std), <@\addr{"y"}@>)

  # predict the rotation
  r_mu = exp(outputs[5])
  r_std = exp(outputs[6])
  @rand(normal(r_mu, r_std), <@\addr{"angle"}@>)

  # predict the size 
  size_alpha = exp(outputs[7])
  size_beta = exp(outputs[8])
  @rand(Gen.beta(size_alpha, size_beta), <@\addr{"size"}@>)
  
  # predict the identity of the letter
  log_letter_dist = outputs[9:end]
  letter_dist = exp.(log_letter_dist)
  letter_dist = letter_dist / sum(letter_dist)
  @rand(categorical(letter_dist), <@\addr{"letter"}@>)
end

proposal = @gen function ()

  # get image from input trace
  image = zeros(1, num_input)
  image[1,:] = @read(<@\addr{"image"}@>)[:]

  # run inference network
  outputs = @tf_call(network(image))

  # make prediction given inference network outputs
  @call(predict(outputs[1,:]), <@\addr{"prediction"}@>)
end

proposal_batch = @gen function (batch_size)

  # get images from input trace
  images = zeros(Float32, batch_size, num_input)
  for i=1:batch_size
    images[i,:] = @read((<@\addr{"\$i"}@>, <@\addr{"image"}@>))[:]
  end

  # run inference network in batch
  outputs = @tf_call(network(images))
  
  # make prediction for each image,
  # given inference network outputs
  for i=1:batch_size
    @call(predict(outputs[i,:]), <@\addr{"\$i"}@>)
  end
end
\end{lstlisting}
\end{minipage}
\caption{}
\label{fig:proposal-code-figure}
\end{figure}


%x = width * @rand(uniform_cont(0, 1), <@\addr{"x"}@>)

\begin{figure}[t]
\begin{subfigure}[b]{0.6\textwidth}
\begin{lstlisting}
grads_and_vars = []
zero_grad_ops = []
for (param_name) in <@\infr{get\_param\_names}@>(network)
  grad = tf.negative(<@\infr{get\_param\_grad}@>(network, param_name))
  var = <@\infr{get\_param\_var}@>(network, param_name)
  push!(grads_and_vars, (grad, var))
  push!(zero_grad_ops, <@\infr{get\_zero\_grad\_op}@>(network, param_name))
end
optimizer = tf.train.AdamOptimizer(1e-4)
network_update = tf.group(
  tf.train.apply_gradients(optimizer, grads_and_vars),
  zero_grad_ops...)
\end{lstlisting}
\caption{Defining the update to the TensorFlow parameters}
\end{subfigure}%
\begin{subfigure}[b]{0.4\textwidth}
\begin{lstlisting}
num_train = 100000
traces = Vector{Trace}(num_train)
for i=1:num_train
  (traces[i], _, _) = <@\infr{simulate}@>(model, ())
end
\end{lstlisting}
\caption{Simulating training data from the model}
\end{subfigure}
\begin{subfigure}[b]{\textwidth}
\begin{lstlisting}
tf.run(get_tf_session(), tf.global_variables_initializer())
batch_size = 100
for iter=1:num_iter
  batch_idx = randperm(num_train)[1:batch_size]
  traces = all_traces[batch_idx]
  vector_trace = <@\infr{vectorize}@>(traces)
  (total_score, _) = <@\infr{backprop}@>(proposal_batch, (batch_size,), vector_trace, vector_trace)
  tf.run(get_tf_session(), network_update)
  score = total_score / batch_size
end
\end{lstlisting}
\caption{Training the proposal on simulated data}
\end{subfigure}
\caption{
Julia code for training the data-driven proposal distribution of Figure~\ref{fig:proposal-code-figure} to propose the latent variables of the generative model (\texttt{model}) of Figure~\ref{fig:model-code-figure} given an observed image.
In (a), we define an TensorFlow (TF) operation (\texttt{network\_update}) that will be used to update the parameters of the TF function \texttt{network} (defined in Figure~\ref{fig:proposal-code-figure}.
We define the operation in terms of the parameter value variables and parameter gradient accumulator variables that are accessible with GenLite API functions \texttt{get\_param\_var} and \texttt{get\_param\_grad}, respectively.
The update applies an ADAM update to all of the parameters and then zeros-out the gradient accumulators.
Next, in (b), we generate training data by sampling traces from the model.
These traces contain both the observed image (at address \texttt{"image"}) and all of the latent variables (\texttt{"x"}, \texttt{"y"}, ..).
Finally, in (c), we perform training using batches.
We group a set of traces of \texttt{model} into a vector-shaped trace using \texttt{vectorize}.
We then run \texttt{backprop} on the generative function \texttt{proposal\_batch}, where we use \texttt{vector\_trace} as both the input trace (from which the image is read) and the output trace (which contains the ground truth latent variables for the corresponding image).
GenLite API functions are shown in purple.
}
\label{fig:training-code-figure}
\end{figure}

\begin{figure}[t]
\begin{minipage}[t]{0.55\textwidth}
\begin{lstlisting}
function importance_sample(input_image, num_samples)
  observations = Trace()
  observations[<@\addr{"image"}@>] = input_image

  # obtain importance samples and their weights
  traces = Vector{Trace}(num_samples)
  log_weights = Vector{Float64}(num_samples)
  for i=1:num_samples
    (traces[i], _, log_weights[i], _) = <@\infr{importance2}@>(
      model, (), proposal, (), observations)
  end

  # return trace with probability proportional to its weight
  probs = exp.(log_weights - logsumexp(log_weights))
  idx = rand(categorical, (probs,))
  return traces[idx]
end
\end{lstlisting}
\end{minipage}
\caption{
Performing inference in the generative model of Figure~\ref{fig:model-code-figure} using a combination of deep learning and model-based Monte Carlo.
Having trained the proposal distribution (\texttt{proposal}) we use it as an importance distribution in a sampling-importance-resampling algorithm.
The GenLite API function \infr{\texttt{importance2}} samples from an importance distribution by simulating from a generative function, and then treats the resulting trace as a proposed trace for the model.
}
\label{fig:inference-code-figure}
\end{figure}


