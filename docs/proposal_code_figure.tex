\begin{figure}[t]
\begin{minipage}[t]{0.5\textwidth}
\begin{lstlisting}
using GenLiteTF
using TensorFlow
tf = TensorFlow

num_input = width * height
num_output = 11

function conv2d(x, W)
  tf.nn.conv2d(x, W, [1, 1, 1, 1], "SAME")
end

function max_pool_2x2(x)
  tf.nn.max_pool(x, [1, 2, 2, 1], [1, 2, 2, 1], "SAME")
end

function initial_weight(shape)
  randn(Float32, shape...) * 0.001f0
end

function initial_bias(shape)
  fill(0.1f0, shape...)
end

network = @tf_function begin

  # input image (N, 56 * 56)
  @input image_flat Float32 [-1, num_input]
  image = tf.reshape(image_flat, [-1, width, height, 1])

  # convolution + max-pooling (N, 28, 28, 32)
  @param W_conv1 initial_weight([5, 5, 1, 32])
  @param b_conv1 initial_bias([32])
  h_conv1 = tf.nn.relu(conv2d(image, W_conv1) + b_conv1)
  h_pool1 = max_pool_2x2(h_conv1)

  # convolution + max-pooling (N, 14, 14, 32)
  @param W_conv2 initial_weight([5, 5, 32, 32])
  @param b_conv2 initial_bias([32])
  h_conv2 = tf.nn.relu(conv2d(h_pool1, W_conv2) + b_conv2)
  h_pool2 = max_pool_2x2(h_conv2)
  h_pool2_flat = tf.reshape(h_pool2, [-1, 14 * 14 * 32])

  # convolution + max-pooling (N, 7, 7, 64)
  @param W_conv3 initial_weight([5, 5, 32, 64])
  @param b_conv3 initial_bias([64])
  h_conv3 = tf.nn.relu(conv2d(h_pool2, W_conv3) + b_conv3)
  h_pool3 = max_pool_2x2(h_conv3)
  h_pool3_flat = tf.reshape(h_pool3, [-1, 7 * 7 * 64])

  # fully connected layer (N, 1024)
  @param W_fc1 initial_weight([7 * 7 * 64, 1024])
  @param b_fc1 initial_bias([1024])
  h_fc1 = tf.nn.relu(h_pool3_flat * W_fc1 + b_fc1)

  # output layer (N, 11)
  @param W_fc2 initial_weight([1024, num_output])
  @param b_fc2 initial_bias([num_output])
  @output Float32 (tf.matmul(h_fc1, W_fc2) + b_fc2)
end

\end{lstlisting}
\end{minipage}%
\hfill
\begin{minipage}[t]{0.4\textwidth}
\begin{lstlisting}
predict = @gen function (outputs)

  # predict the x-coordinate
  x_mu = outputs[1]
  x_std = exp(outputs[2])
  @rand(normal(x_mu, x_std), <@\addr{"x"}@>)

  # predict the y-coordinate
  y_mu = outputs[3]
  y_std = exp(outputs[4])
  @rand(normal(y_mu, y_std), <@\addr{"y"}@>)

  # predict the rotation
  r_mu = exp(outputs[5])
  r_std = exp(outputs[6])
  @rand(normal(r_mu, r_std), <@\addr{"angle"}@>)

  # predict the size 
  size_alpha = exp(outputs[7])
  size_beta = exp(outputs[8])
  @rand(Gen.beta(size_alpha, size_beta), <@\addr{"size"}@>)
  
  # predict the identity of the letter
  log_letter_dist = outputs[9:end]
  letter_dist = exp.(log_letter_dist)
  letter_dist = letter_dist / sum(letter_dist)
  @rand(categorical(letter_dist), <@\addr{"letter"}@>)
end

proposal = @gen function ()

  # get image from input trace
  image = zeros(1, num_input)
  image[1,:] = @read(<@\addr{"image"}@>)[:]

  # run inference network
  outputs = @tf_call(network(image))

  # make prediction given inference network outputs
  @call(predict(outputs[1,:]), <@\addr{"prediction"}@>)
end

proposal_batch = @gen function (batch_size)

  # get images from input trace
  images = zeros(Float32, batch_size, num_input)
  for i=1:batch_size
    images[i,:] = @read((<@\addr{"\$i"}@>, <@\addr{"image"}@>))[:]
  end

  # run inference network in batch
  outputs = @tf_call(network(images))
  
  # make prediction for each image
  for i=1:batch_size
    @call(predict(outputs[i,:]), <@\addr{"\$i"}@>)
  end
end
\end{lstlisting}
\end{minipage}
\caption{
A GenLite TensorFlow (TF) function (\texttt{network}) that is invoked by a generative function (\texttt{proposal}) that implements a data-driven proposal for the model of Figure~\ref{fig:model-code-figure}.
GenLite TF functions are identified by a \texttt{@tf\_function} keyword.
The user declares inputs (\texttt{@input}, corresponding to TF placeholders), parameters (\texttt{@param}, corresponding to TF variables) and an output tensor (\texttt{@output}).
The rest of the code inside the \texttt{@tf\_function} block is regular TensorFlow code, using the TensorFlow.jl Julia wrapper \cite{?} around the TensorFlow C API.
The proposal reads the image from an input trace, runs the network, and then uses its output to parametrize distributions on the latent variables in the model.
To scalably train the network on GPU hardware, we also implement a batched variant of the proposal program, which reads from and writes to vector-shaped traces.
The batched and unbatched variants of the reuse both the TensorFlow and probabilistic prediction code.
}
\label{fig:proposal-code-figure}
\end{figure}


%x = width * @rand(uniform_cont(0, 1), <@\addr{"x"}@>)
