\documentclass{article}

\PassOptionsToPackage{numbers, compress}{natbib}
\usepackage{nips_2017}
\usepackage{listings}
\usepackage{textcomp}
\usepackage[usenames,dvipsnames,svgnames]{xcolor}
\usepackage{graphicx}

%!TEX root = ./main.tex

\lstset{
  basicstyle=\ttfamily\scriptsize,
  columns=fullflexible,
  keepspaces=true,
  upquote=true,
  % Define . and % and @ as letters to include them in keywords.
  alsoletter={\.,\%,\#, \@, \?, \/},
  % First type of keywords.
  morekeywords=[1]{function, if, else, end, for, begin, in, const, struct, using, return, const},
  keywordstyle=[1]\textcolor{Brown},
  % Second type of keywords.
  morekeywords=[2]{\@gen, \@rand, \@param, \@input, \@output, \@tf_function, \@call, \@read, \@tf_call},
  keywordstyle=[2]\textcolor{RoyalBlue},
  % Add strings
  showstringspaces=False,
  %stringstyle=\ttfamily\color{NavyBlue},
  %stringstyle=\ttfamily\color{Purple},
  %morestring=[b]{"},
  %morestring=[b]{'},
  % l is for line comment
  morecomment=[l]{\#},
  commentstyle=\color{Gray}\ttfamily,
  escapeinside={<@}{@>}
}

\newcommand{\addr}[1]{\textcolor{DarkGreen}{#1}}
\newcommand{\infr}[1]{\textcolor{Purple}{#1}}


\title{Combining custom deep learning and Monte Carlo for inference in probabilistic programs}

\begin{document}

\maketitle

\begin{abstract}

\end{abstract}

\section{Background: Programmable Inference in Probabilistic Programs}
Introduce GenLite.

\subsection{Generative Functions}
Introduce generative functions.

\subsection{Using Generative Functions to Define Proposal Distributions}
Like probabilistic models, we represent proposals as probabilistic programs.
Proposal programs can be used in importance sampling, sequential Monte Carlo, and Markov chain Monte Carlo.
Proposal programs can include latent variables.

\subsection{Training Proposal Distributions on Simulated Data}
Proposal programs can be used on simulated data (cite the `probabilitsic programs as proposals' research).
Proposal programs can be trained for use as importance distributions or MCMC proposals.
Show the math for training (KL divergence and maximum likelihood).
Discuss static parameters.

\section{Scalable Deep Learning in GenLite using TensorFlow}
About the implementation.
GenLite is embedded in Julia.
TensorFlow functions, which have inputs and outputs and parameters.
Discuss the reverse-mode AD integration.
Discuss vectorized (batched) training.

\begin{figure}[h]
\centering
    \includegraphics[width=0.7\textwidth]{images/tf-integration-schematic.pdf}
    \caption{Reverse-mode AD in GenLite integrated with TensorFlow.}
    \label{fig:tf-integration-schematic}
\end{figure}

\section{Combining Monte Carlo Inference and Deep Learning Proposals}
Because GenLite provides programmable inference, we can combine deep neural network proposals with other Monte Carlo strategies, like random walk moves, for refining a hypothesis.

\begin{figure}[t]
\begin{minipage}[t]{0.5\textwidth}
\begin{lstlisting}
using Cairo, ImageMagick, ImageFiltering
..

function render(glyph::Glyph)
  canvas = CairoRGBSurface(width, height)
  cc = CairoContext(canvas)
  Cairo.save(cc)

  # set background color to white
  Cairo.set_source_rgb(cc, 1.0, 1.0, 1.0)
  Cairo.rectangle(cc, 0.0, 0.0, width, height)
  Cairo.fill(cc)
  Cairo.restore(cc)
  Cairo.save(cc)

  # write the letter
  fontface = "Sans $(glyph.fontsize)"
  Cairo.set_font_face(cc, fontface)
  Cairo.text(cc, glyph.x, glyph.y, glyph.letter,
             angle=glyph.angle)

  return convert_to_png_blob(canvas)
end
\end{lstlisting}
\end{minipage}%
\begin{minipage}[t]{0.5\textwidth}
\begin{lstlisting}
model = @gen function()
  # prior
  x = width * @rand(uniform_cont(0, 1), <@\addr{"x"}@>)
  y = height * @rand(uniform_cont(0, 1), <@\addr{"y"}@>)
  size = @rand(uniform_cont(0, 1), <@\addr{"size"}@>)
  letter_id = @rand(uniform_disc(1, 3), <@\addr{"letter"}@>)
  letter = ["A", "B", "C"][letter_id]
  angle = 45 * @rand(uniform_cont(-1, 1), <@\addr{"angle"}@>)
  fontsize = scale_size(min_size, max_size, size)
  glyph = Glyph(x, y, angle, fontsize, letter)

  # render to png bytes
  image_png = render(glyph)

  # add Gaussian blur
  blur_width = 3
  blurred_png = imfilter(image_png,
                  Kernel.gaussian(blur_width))

  # add noise
  matrix = convert(Matrix{Float64}, blurred_png)
  @rand(speckle_noise(matrix, 0.1), <@\addr{"image"}@>)
end
\end{lstlisting}
\end{minipage}
\caption{Generative function for a generative model of blurry images that contain a single letter at a random location, rotation, and size. Addresses of random choices are shown in green.}
\label{fig:model-code-figure}
\end{figure}

\begin{figure}[t]
\begin{minipage}[t]{0.6\textwidth}
\begin{lstlisting}
using GenLiteTF
using TensorFlow
tf = TensorFlow

num_input = width * height
num_output = 11

function conv2d(x, W)
  tf.nn.conv2d(x, W, [1, 1, 1, 1], "SAME")
end

function max_pool_2x2(x)
  tf.nn.max_pool(x, [1, 2, 2, 1], [1, 2, 2, 1], "SAME")
end

function initial_weight(shape)
  randn(Float32, shape...) * 0.001f0
end

function initial_bias(shape)
  fill(0.1f0, shape...)
end

network = @tf_function begin

  # input image (N, 56 * 56)
  @input image_flat Float32 [-1, num_input]
  image = tf.reshape(image_flat, [-1, width, height, 1])

  # convolution + max-pooling (N, 28, 28, 32)
  @param W_conv1 initial_weight([5, 5, 1, 32])
  @param b_conv1 initial_bias([32])
  h_conv1 = tf.nn.relu(conv2d(image, W_conv1) + b_conv1)
  h_pool1 = max_pool_2x2(h_conv1)

  # convolution + max-pooling (N, 14, 14, 32)
  @param W_conv2 initial_weight([5, 5, 32, 32])
  @param b_conv2 initial_bias([32])
  h_conv2 = tf.nn.relu(conv2d(h_pool1, W_conv2) + b_conv2)
  h_pool2 = max_pool_2x2(h_conv2)
  h_pool2_flat = tf.reshape(h_pool2, [-1, 14 * 14 * 32])

  # convolution + max-pooling (N, 7, 7, 64)
  @param W_conv3 initial_weight([5, 5, 32, 64])
  @param b_conv3 initial_bias([64])
  h_conv3 = tf.nn.relu(conv2d(h_pool2, W_conv3) + b_conv3)
  h_pool3 = max_pool_2x2(h_conv3)
  h_pool3_flat = tf.reshape(h_pool3, [-1, 7 * 7 * 64])

  # fully connected layer (N, 1024)
  @param W_fc1 initial_weight([7 * 7 * 64, 1024])
  @param b_fc1 initial_bias([1024])
  h_fc1 = tf.nn.relu(h_pool3_flat * W_fc1 + b_fc1)

  # output layer (N, 11)
  @param W_fc2 initial_weight([1024, num_output])
  @param b_fc2 initial_bias([num_output])
  @output Float32 (tf.matmul(h_fc1, W_fc2) + b_fc2)
end

\end{lstlisting}
\end{minipage}%
\begin{minipage}[t]{0.4\textwidth}
\begin{lstlisting}
predict = @gen function (outputs)

  # predict the x-coordinate
  x_mu = outputs[1]
  x_std = exp(outputs[2])
  @rand(normal(x_mu, x_std), <@\addr{"x"}@>)

  # predict the y-coordinate
  y_mu = outputs[3]
  y_std = exp(outputs[4])
  @rand(normal(y_mu, y_std), <@\addr{"y"}@>)

  # predict the rotation
  r_mu = exp(outputs[5])
  r_std = exp(outputs[6])
  @rand(normal(r_mu, r_std), <@\addr{"angle"}@>)

  # predict the size 
  size_alpha = exp(outputs[7])
  size_beta = exp(outputs[8])
  @rand(Gen.beta(size_alpha, size_beta), <@\addr{"size"}@>)
  
  # predict the identity of the letter
  log_letter_dist = outputs[9:end]
  letter_dist = exp.(log_letter_dist)
  letter_dist = letter_dist / sum(letter_dist)
  @rand(categorical(letter_dist), <@\addr{"letter"}@>)
end

proposal = @gen function ()

  # get image from input trace
  image = zeros(1, num_input)
  image[1,:] = @read(<@\addr{"image"}@>)[:]

  # run inference network
  outputs = @tf_call(network(image))

  # make prediction given inference network outputs
  @call(predict(outputs[1,:]), <@\addr{"prediction"}@>)
end

proposal_batch = @gen function (batch_size)

  # get images from input trace
  images = zeros(Float32, batch_size, num_input)
  for i=1:batch_size
    images[i,:] = @read((<@\addr{"\$i"}@>, <@\addr{"image"}@>))[:]
  end

  # run inference network in batch
  outputs = @tf_call(network(images))
  
  # make prediction for each image,
  # given inference network outputs
  for i=1:batch_size
    @call(predict(outputs[i,:]), <@\addr{"\$i"}@>)
  end
end
\end{lstlisting}
\end{minipage}
\caption{}
\label{fig:proposal-code-figure}
\end{figure}


%x = width * @rand(uniform_cont(0, 1), <@\addr{"x"}@>)


\section{Example}
See Figure~\ref{fig:model-code-figure}.

\section{Related Work}
Guide programs in Pyro,
Using probabilisic programs as proposals,
Edward,
Stuart Russell work on block neural proposals,
Tuan An Le's work on univeral compiled inference,
Wake sleep,
Helmholtz machines,
VAE,
Stochastic inverses

%\begin{abstract}
Probabilistic programming provides concise and flexible languages for generative modeling, but achieving reliably efficient inference in probabilistic programming systems based on automated Monte Carlo algorithms remains challenging.
Recently, user-programmable inference and offline training of proposal distributions based on deep neural networks have been proposed as two solutions to this problem.
This paper proposes to combine these two approaches by permitting users to define parametrized proposal distributions as probabilistic programs based on deep neural networks.
Users train these networks on data simulated from the generative model, which is also represented as a probabilistic program.
The paper demonstrates the approach with models and inference programs written in GenLite, a probabilistic programming language with user-programmable inference that uses TensorFlow for scalable deep learning.
\end{abstract}

%\input{introduction.tex}
%\input{background.tex}
%\input{formalism.tex}
%\section{Related Work}
\begin{itemize}
\item Guide programs in Pyro
\item Using probabilisic programs as proposals \cite{cusumano2018using}
\item Edward \cite{tran2016edward}
\item Stuart Russell work on block neural proposals \cite{wang2017neural}
\item Univeral compiled inference in probabilistic programs \cite{le2016inference}
\item Stochastic inverses \cite{stuhlmuller2013learning}
\item Wake sleep, Helmholtz machines, VAE
\item Other work on data-driven proposals (e.g. \cite{tu2002image}).
\end{itemize}

%\input{experiments.tex}
%\input{discussion}

%\subsubsection*{Acknowledgments}
%This research was supported by DARPA (PPAML program, contract number FA8750-14-2-0004), IARPA (under research contract 2015-15061000003), the Office of Naval Research (under research contract N000141310333), the Army Research Office (under agreement number W911NF-13-1-0212), and gifts from Analog Devices and Google.
%This research was conducted with Government support under and awarded by DoD, Air Force Office of Scientific Research, National Defense Science and Engineering Graduate (NDSEG) Fellowship, 32 CFR 168a.

\bibliographystyle{unsrtnat}
\bibliography{references}

\end{document}
